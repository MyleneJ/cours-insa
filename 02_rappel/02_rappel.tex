\documentclass[aspectratio=169,obeyspaces,spaces,hyphens,dvipsnames]{beamer}
\usepackage[utf8x]{inputenc}
\usepackage{lmodern}% http://ctan.org/pkg/lm
\usepackage{minted}
\usepackage{hyperref}
\usepackage{xcolor}
\usepackage{pgfplots}
\usepackage{tikz}
\usepackage[normalem]{ulem}
\usepackage{textcomp}
\usepackage{marvosym} % \MVRIGHTarrow
\usepackage{tabularx}

\mode<presentation>
\usetheme{INSA}

\def\signed #1{{\leavevmode\unskip\nobreak\hfil\penalty50\hskip2em
  \hbox{}\nobreak\hfil(#1)
  \parfillskip=0pt \finalhyphendemerits=0 \endgraf}}

\newsavebox\mybox
\newenvironment{aquote}[1]
  {\savebox\mybox{#1}\begin{quotation}}
  {\signed{\usebox\mybox}\end{quotation}}

\authors{Mylène Josserand}
\conference{Cours INSA}
\email{josserand.mylene@gmail.com}
\slidesurl{}

\title{Introduction à Linux embarqué - quelques rappels}
\author[Mylène Josserand]
{Mylène Josserand}
\date[Octobre 2019]
{Cours INSA, Octobre 2019 \\
  \vspace{0.5cm}
  \includegraphics[scale=0.1]{pictures/insa-tls.png}
  \hspace{0.5cm}
  \includegraphics[scale=0.1]{pictures/tux.png}
}
\institute[]
{Développeuse et formatrice Linux embarqué}

\begin{document}

\begin{frame}
  \titlepage
\end{frame}

\begin{frame}{Qu'est-ce que Linux embarqué?}
  \huge
  \begin{center}
    Linux Embarqué est l'utilisation du {\bf kernel Linux} et différents
    composants {\bf open-source}(GNU) dans des {\bf systèmes embarqués}
  \end{center}
\end{frame}

\begin{frame}
  \frametitle{Architecture globale}
  \begin{center}
    \includegraphics[width=0.9\textwidth]{graphics/02_rappel/global-architecture.pdf}
  \end{center}
\end{frame}

\begin{frame}
  \frametitle{Composants software}
  \begin{itemize}
  \item Toolchain de cross-compilation
    \begin{itemize}
    \item Compilateur qui est executé sur la machine de développement
      mais qui génère du code pour la cible
    \end{itemize}
  \item Bootloader
    \begin{itemize}
    \item Démarre le hardware, responsable des initialisations basiques,
      charge et execute le kernel
    \end{itemize}
  \item Kernel Linux
    \begin{itemize}
    \item Contient la gestion des processus et de la mémoire, le réseau, les
      pilotes des périphériques et fournit des services aux applications
      user space
    \end{itemize}
  \item Librairie C
    \begin{itemize}
    \item Interface entre le kernel et les applications user space
    \end{itemize}
  \item Bibliothèques et applications
    \begin{itemize}
    \item De tiers ou faites maisons
    \end{itemize}
  \end{itemize}
\end{frame}


\begin{frame}
  \frametitle{Kernel Linux dans le système}
  \begin{center}
    \includegraphics[height=0.8\textheight]{graphics/02_rappel/linux-kernel-in-system.pdf}
  \end{center}
\end{frame}


\begin{frame}{Dans le kernel Linux}
  \begin{center}
    \includegraphics[width=\textwidth]{graphics/02_rappel/inside-linux-kernel.pdf}
  \end{center}
\end{frame}

\begin{frame}
  \frametitle{Cross-toolchain}
  \begin{center}
    \includegraphics[width=0.8\textwidth]{graphics/02_rappel/cross-toolchain.pdf}
  \end{center}
\end{frame}

\begin{frame}
  \frametitle{Differentes procédures de compilation}
  \begin{center}
    \includegraphics[height=0.8\textheight]{graphics/02_rappel/toolchain-build-types.pdf}
  \end{center}
\end{frame}

\begin{frame}
  \frametitle{Composants d'une toolchain}
  \begin{center}
    \includegraphics[width=0.8\textwidth]{graphics/02_rappel/components.pdf}
  \end{center}
\end{frame}

\begin{frame}
  \begin{center}
    \Huge
    Merci de votre attention ! \\
    Questions ? Commentaires ?\\
    \vspace{1cm}
    \large
    Mylène Josserand — \code{josserand.mylene@gmail.com}\\
    \vspace{1cm}
    Slides under CC-BY-SA 3.0\\
    \scriptsize{© Copyright 2004-\the\year, Bootlin\\
    \url{https://github.com/MyleneJ/cours-insa}}
  \end{center}
\end{frame}

\end{document}
