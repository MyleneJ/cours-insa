\documentclass[aspectratio=169,obeyspaces,spaces,hyphens,dvipsnames]{beamer}
\usepackage[utf8x]{inputenc}
\usepackage{lmodern}% http://ctan.org/pkg/lm
\usepackage{minted}
\usepackage{hyperref}
\usepackage{xcolor}
\usepackage{pgfplots}
\usepackage{tikz}
\usepackage[normalem]{ulem}
\usepackage{textcomp}
\usepackage{marvosym} % \MVRIGHTarrow
\usepackage{tabularx}

\mode<presentation>
\usetheme{INSA}

\def\signed #1{{\leavevmode\unskip\nobreak\hfil\penalty50\hskip2em
  \hbox{}\nobreak\hfil(#1)
  \parfillskip=0pt \finalhyphendemerits=0 \endgraf}}

\newsavebox\mybox
\newenvironment{aquote}[1]
  {\savebox\mybox{#1}\begin{quotation}}
  {\signed{\usebox\mybox}\end{quotation}}

\authors{Mylène Josserand}
\conference{Cours INSA}
\email{josserand.mylene@gmail.com}
\slidesurl{}

\title{Introduction à Linux embarqué - Buildroot}
\author[Mylène Josserand]
{Mylène Josserand}
\date[Octobre 2019]
{Cours INSA, Octobre 2019 \\
  \vspace{0.5cm}
  \includegraphics[scale=0.1]{pictures/insa-tls.png}
  \hspace{0.5cm}
  \includegraphics[scale=0.1]{pictures/br.png}
}
\institute[]
{Développeuse et formatrice Linux embarqué}

\begin{document}

\begin{frame}
  \titlepage
\end{frame}

%%%%%%%%%%%%%%%%%%%%%%%%%%%%%%%%%%%%%%%%%
\section{Configuration \& compilation}
%%%%%%%%%%%%%%%%%%%%%%%%%%%%%%%%%%%%%%%%%

\begin{frame}{Organisation du build}
  \begin{itemize}
  \item Toutes les sorties de compilation vont dans \code{output/}
  \item Le fichier de configuration est un \code{.config}
  \item \code{buildroot/}
    \begin{itemize}
    \item {\bf \code{.config}}
    \item \code{arch/}
    \item \code{package/}
    \item {\bf \code{output/}}
    \item \code{fs/}
    \item ...
    \end{itemize}
  \end{itemize}
\end{frame}

\begin{frame}{Config entière vs. {\em defconfig}}
  \begin{itemize}
  \item \code{.config} = fichier de config {\em entier}
    \MVRightarrow contient les valeurs de toutes les options \\
    Le \code{.config} par défaut a +3000 lignes \\
    \MVRightarrow Pas pratique et lisible
  \item Un \code{defconfig} stocke seulement les options qui n'ont pas de
    valeurs par défaut \\
    \MVRightarrow Plus facile à lire et modifiable par des humains
  \end{itemize}
\end{frame}

\begin{frame}[fragile]{{\em defconfig}: exemple}
  \begin{itemize}
  \item La configuration Buildroot par défaut, le {\em defconfig}
    est vide: tout est par défaut.
  \item Si change pour ARM, le {\em defconfig} aura juste une ligne :
{\small
\begin{block}{}
\begin{verbatim}
BR2_arm=y
\end{verbatim}
\end{block}
}
\item Si ajoute en plus le paquet \code{stress}, le {\em defconfig}
  sera deux lignes :
{\small
\begin{block}{}
\begin{verbatim}
BR2_arm=y
BR2_PACKAGE_STRESS=y
\end{verbatim}
\end{block}
}
  \end{itemize}
\end{frame}

\begin{frame}{Utilisation et création d'un {\em defconfig}}
  \begin{itemize}
  \item Buildroot permet de charger un {\em defconfig} issu du dossier
    \code{configs/} via :
    \code{make <foo>_defconfig}
    \begin{itemize}
    \item Cela va surcharger le \code{.config}, si il y en a un
    \end{itemize}
  \item Pour créer un {\em defconfig} :
    \code{make savedefconfig}
    \begin{itemize}
    \item Fichier pointé par l'option \code{BR2_DEFCONFIG}
    \item Par défaut, pointe sur le \code{defconfig} original
    \end{itemize}
  \end{itemize}
\end{frame}

\begin{frame}[fragile]{{\em defconfigs} existants}
  \begin{itemize}
  \item  Il existe un {\em defconfigs} pour différentes plateformes supportées :
    \begin{itemize}
    \item RaspberryPi, BeagleBone Black, CubieBoard, Microchip evaluation
      boards, Minnowboard, quelques i.MX6 boards
    \end{itemize}
  \item Pour les lister : \code{make list-defconfigs}
  \item La plupart des {\em defconfigs} sont minimaux : compile seulement une
    toolchain, un bootloader, un kernel et un root filesystem minimal.
    \begin{block}{}
\begin{verbatim}
$ make qemu_arm_vexpress_defconfig
$ make
\end{verbatim}
    \end{block}
  \item Des instructions sont souvent disponibles dans
    \code{board/<boardname>}, e.g.: \code{board/raspberrypi/readme.txt}.
  \end{itemize}
\end{frame}

\begin{frame}[fragile]{Quelques tips}
  \begin{itemize}
  \item Nettoyer une cible
    \begin{itemize}
    \item Nettoyer tout le dossier d'output mais garder la configuration :
      \begin{block}{}
\begin{verbatim}
$ make clean
\end{verbatim}
      \end{block}
    \item Tout nettoyer dont configuration et téléchargement des sources :
      \begin{block}{}
\begin{verbatim}
$ make distclean
\end{verbatim}
      \end{block}
    \end{itemize}
  \item Pour avoir une compilation plus verbose
    \begin{itemize}
    \item Par défaut, beaucoup de commandes sont cachées
    \item \code{V=1} : pour avoir toutes les commandes
      \begin{block}{}
\begin{verbatim}
$ make V=1
\end{verbatim}
      \end{block}
    \end{itemize}
  \end{itemize}
\end{frame}

%%%%%%%%%%%%%%%%%%%%%%%%%%%%%%%%%%%%%%%%%
\section{Buildroot organisation (source/build)}
%%%%%%%%%%%%%%%%%%%%%%%%%%%%%%%%%%%%%%%%%

\subsection{Source tree}

\begin{frame}{Source tree (1/3)}
  \begin{itemize}
  \item \code{Makefile}
    \begin{itemize}
    \item top-level \code{Makefile}, gère la configuration et orchestre le build
    \end{itemize}
  \item \code{Config.in}
    \begin{itemize}
    \item top-level \code{Config.in}, options générales + inclusion de tous les
      autres \code{Config.in}
    \end{itemize}
  \end{itemize}
\end{frame}

\begin{frame}{Source tree (2/3)}
  \begin{itemize}
  \item \code{toolchain/} : paquets pour générer ou utiliser une toolchain
  \item \code{system/}
    \begin{itemize}
    \item \code{skeleton/} le squelette du rootfs
    \item \code{Config.in}, options pour des fonctionnalités système comme
      le système de démarrage, etc.
    \end{itemize}
  \item \code{linux/} : le paquet du kernel Linux
  \item \code{package/} : tous les paquets user space (2200+)
  \item \code{fs/} : Logique pour générer une image rootfs dans différents
    formats (\code{ext2}, \code{squashfs}, \code{tar}, \code{ubifs}, etc)
  \item \code{boot/} : paquets de bootloader (\code{barebox}, \code{uboot}, etc)
  \end{itemize}
\end{frame}

\begin{frame}{Source tree (3/3)}
  \begin{itemize}
  \item \code{configs/}
    \begin{itemize}
    \item configurations par défaut pour de nombreuses plateformes
    \item Similaire aux defconfigs du kernel
    \item \code{atmel_xplained_defconfig},
      \code{beaglebone_defconfig}, \code{raspberrypi_defconfig}, etc.
    \end{itemize}
  \item \code{board/}
    \begin{itemize}
    \item Fichiers sépcifiques aux boards (config kernel, patches kernel, ...)
    \item Vont de paire avec un {\em defconfig} dans \code{configs/}
    \end{itemize}
  \item \code{utils/} : des utilitaires pour des developpeurs Buildroot
  \item \code{docs/} : documentation en AsciiDoc pouvant être de différents
    formats \url{https://buildroot.org/downloads/manual/manual.html}
  \end{itemize}
\end{frame}

\subsection{Build tree}

\begin{frame}{Build tree}
  \begin{itemize}
  \item \code{output/}
  \item Répertoire global
    \begin{itemize}
    \item \code{build/}
      \begin{itemize}
        \tiny
      \item \code{buildroot-config/}
      \item \code{busybox-1.22.1/}
      \item \code{host-pkgconf-0.8.9/}
      \item \code{kmod-1.18/}
      \item \code{build-time.log}
      \end{itemize}
    \item Lieu d'extraction de toutes les archives
    \item Lieu de compilation de chaque paquet
    \item Ajout en plus de fichiers de {\em stamp} créés par BR
    \item Variable: \code{BUILD_DIR}
    \end{itemize}
  \end{itemize}
\end{frame}

\begin{frame}{Build tree: {\tt output/host}}
  \begin{itemize}
  \item \code{output/}
    \begin{itemize}
    \item \code{host/}
      \begin{itemize}
        \tiny
      \item \code{lib}
      \item \code{bin}
      \item \code{sbin}
        \vspace{0.2cm}
      \item \code{<tuple>/sysroot/bin}
      \item \code{<tuple>/sysroot/lib}
      \item \code{<tuple>/sysroot/usr/lib}
      \item \code{<tuple>/sysroot/usr/bin}
      \end{itemize}
    \item Contient des outils pour l'hôte (cross-compiler, etc.) et le
      {\em sysroot} pour la toolchain
    \item Variable: \code{HOST_DIR}
    \item Variable for the {\em sysroot}: \code{STAGING_DIR}
    \item \code{<tuple>} est un identifiant pour l'architecture, le vendeur,
      l'os, etc : \code{arm-unknown-linux-gnueabihf}.
    \end{itemize}
  \end{itemize}
\end{frame}

\begin{frame}{Build tree: {\tt output/target}}
  \begin{itemize}
  \item \code{output/}
    \begin{itemize}
    \item \code{target/}
      \begin{itemize}
        \tiny
      \item \code{bin/}
      \item \code{etc/}
      \item \code{lib/}
      \item \code{usr/bin/}
      \item \code{usr/lib/}
      \item \code{usr/share/}
      \item \code{usr/sbin/}
      \item \code{THIS_IS_NOT_YOUR_ROOT_FILESYSTEM}
      \item ...
      \end{itemize}
    \item Le rootfs \textbf{cible}
    \item Pas complètement prêt pour la cible: permissions, fichiers device, etc
    \item Utilisé pour générer le rootfs final \code{images/}
    \item Variable: \code{TARGET_DIR}
    \end{itemize}
  \end{itemize}
\end{frame}

\begin{frame}{Build tree: {\tt output/images}}
  \begin{itemize}
  \item \code{output/}
    \begin{itemize}
    \item \code{images/}
      \begin{itemize}
        \scriptsize
      \item \code{zImage}
      \item \code{armada-370-mirabox.dtb}
      \item \code{rootfs.tar}
      \item \code{rootfs.ubi}
      \end{itemize}
    \item Contient les images finales: kernel, bootloader, image(s) rootfs
    \item Variable: \code{BINARIES_DIR}
    \end{itemize}
  \end{itemize}
\end{frame}

%%%%%%%%%%%%%%%%%%%%%%%%%%%%%%%%%%%%%%%%%
\section{Root filesystem dans Buildroot}
%%%%%%%%%%%%%%%%%%%%%%%%%%%%%%%%%%%%%%%%%

\begin{frame}{Construction d'un rootfs}
  \begin{center}
    \includegraphics[height=0.8\textheight]{graphics/03_buildroot_intro/overall-steps.pdf}
  \end{center}
\end{frame}

\begin{frame}[fragile]{Root filesystem overlay}
  \begin{itemize}
  \item  {\bf root filesystem overlay} : Permet de customiser le contenu
    du rootfs : ajout de config, scripts, répertoires, etc
  \item Un {\em root filesystem overlay} est un répertoire dont le contenu sera
    {\bf copié au dessus du root filesystem}. Permet d'écraser des fichiers.
  \item Option \code{BR2_ROOTFS_OVERLAY} avec liste des chemins du rootfs overlay
    \begin{block}{}
      {\small
\begin{verbatim}
$ grep ^BR2_ROOTFS_OVERLAY .config
BR2_ROOTFS_OVERLAY="board/myproject/rootfs-overlay"
$ ls -l board/myproject/rootfs-overlay
board/myproject/rootfs-overlay/etc/ssh/sshd_config
board/myproject/rootfs-overlay/etc/init.d/S99myapp
\end{verbatim}}
      \end{block}
    \end{itemize}
  \end{frame}

\begin{frame}{Scripts Post-build}
  \begin{itemize}
  \item Parfois, le {\em root filesystem overlay} ne suffit pas
    \MVRightarrow {\bf scripts post-build}.
  \item Pour {\bf customiser des fichiers existants}, {\bf supprimer
      des fichiers} pour sauver de l'espace, etc
  \item Executé \textbf{avant} que l'image du rootfs soit créée
  \item \code{BR2_ROOTFS_POST_BUILD_SCRIPT} : liste du chemin de scripts
  \item Contient différents arguments
  \end{itemize}
\end{frame}

\begin{frame}[fragile]{Script post-build: exemple}

\begin{block}{board/myproject/post-build.sh}
\begin{minted}[fontsize=\tiny]{bash}
#!/bin/sh
TARGET_DIR=$1
BOARD_DIR=board/myproject/

# Generate a file identifying the build (git commit and build date)
echo $(git describe) $(date +%Y-%m-%d-%H:%M:%S) > \
    $TARGET_DIR/etc/build-id

# Create /applog mountpoint, and adjust /etc/fstab
mkdir -p $TARGET_DIR/applog
grep -q "^/dev/mtdblock7" $TARGET_DIR/etc/fstab || \
    echo "/dev/mtdblock7\t\t/applog\tjffs2\tdefaults\t\t0\t0" >> \
    $TARGET_DIR/etc/fstab

# Remove unneeded files
rm -rf $TARGET_DIR/usr/share/icons/bar
\end{minted}
\end{block}

\begin{block}{Buildroot configuration}
{\scriptsize
\begin{verbatim}
BR2_ROOTFS_POST_BUILD_SCRIPT="board/myproject/post-build.sh"
\end{verbatim}}
\end{block}

\end{frame}

\begin{frame}{Scripts post-image}
  \begin{itemize}
  \item \code{Script post-image} : Executé \textbf{à la fin} de la génération
    de l'image du rootfs
  \item Pour faire n'importe quelle action nécessaire en fin de build. Exemple:
    \begin{itemize}
    \item Générer une image finale
    \item Démarrer un processus de flashage
    \end{itemize}
  \item \code{BR2_ROOTFS_POST_IMAGE_SCRIPT} : liste de {\em post-image} scripts
  \end{itemize}
\end{frame}

\begin{frame}{Déploiement des images}
  \begin{itemize}
  \item Par défaut, Buildroot stocke les images dans \code{output/images}
  \item Déploiement doit être fait par l'utilisateur
  \item Si stockage amovible comme carte SD ou clef USB :
    \begin{itemize}
    \item Création manuellement des partitions et extraction des composants
    \item Utilisation d'un outil \code{genimage} pour tout gérer
    \end{itemize}
  \end{itemize}
\end{frame}

\begin{frame}{Déploiement des images : genimage}
  \begin{itemize}
  \item \code{genimage} : peut créer des image complètes pour des périphériques
    blocks (carte SD, clef USB, disque dur)
  \item Exemple commun avec 2 partitions : une partition FAT
    pour bootloader \& kernel, une partition ext4 pour le root filesystem
  \item Il existe un script d'aide \code{support/scripts/genimage.sh}
  \item De plus en plus utilisé \MVRightarrow Image toute prête
  \end{itemize}
\end{frame}

\begin{frame}[fragile]{Déploiement des images: exemple genimage}
\begin{block}{genimage-raspberrypi.cfg}
\begin{columns}
  \column{0.4\textwidth}
{\tiny
  \begin{verbatim}
image boot.vfat {
  vfat {
    files = {
      "bcm2708-rpi-b.dtb",
      "rpi-firmware/bootcode.bin",
      "rpi-firmware/cmdline.txt",
      "zImage"
      [...]
    }
  }
}
\end{verbatim}
}
\column{0.4\textwidth}
{\tiny
\begin{verbatim}
image sdcard.img {
  hdimage {
  }
  partition boot {
    partition-type = 0xC
    bootable = "true"
    image = "boot.vfat"
  }
  partition rootfs {
    partition-type = 0x83
    image = "rootfs.ext4"
  }
}
 \end{verbatim}
}
\end{columns}
\end{block}
\begin{block}{defconfig}
{\tiny
  \begin{verbatim}
BR2_ROOTFS_POST_IMAGE_SCRIPT="support/scripts/genimage.sh"
BR2_ROOTFS_POST_SCRIPT_ARGS="-c board/raspberrypi/genimage-raspberrypi.cfg"
\end{verbatim}
}
\end{block}
\begin{block}{flash}
{\tiny
  \begin{verbatim}
dd if=output/images/sdcard.img of=/dev/sdb
\end{verbatim}
}
\end{block}
\end{frame}

\begin{frame}
  \begin{center}
    \Huge
    Merci de votre attention ! \\
    Questions ? Commentaires ?\\
    \vspace{1cm}
    \large
    Mylène Josserand — \code{josserand.mylene@gmail.com}\\
    \vspace{1cm}
    Slides under CC-BY-SA 3.0\\
    \scriptsize{© Copyright 2004-\the\year, Bootlin\\
    \url{https://github.com/MyleneJ/cours-insa}}
  \end{center}
\end{frame}

\end{document}
