\chapter
{Création d'un nouveau paquet dans Buildroot}
{Objectifs:
  \begin{itemize}
  \item Création d'un nouveau paquet pour {\em nInvaders}
  \item Comprendre comment ajouter des dépendances
  \end{itemize}
}

\section{Préparation}

Après une recherche Google, trouver le site web pour {\em nInvaders}
et télécharger le code source. Analyser son système de build et
déduire quelle infrastructure de paquet est la plus appropriée.

\section{Paquet minimal}

Créer un répertoire pour ce paquet dans les sources de Buildroot :
\code{package/ninvaders}. Créer un fichier \code{Config.in} avec une option
pour activer ce paquet et un fichier \code{ninvaders.mk} minimal qui spécifiera
ce qu'il faut pour juste {\em télécharger} le paquet.

Pour information, l'URL de téléchargement des sources {\em nInvaders} est
\code{http://downloads.sourceforge.net/project/ninvaders/ninvaders/0.1.1/}.

Note: Pour y arriver, seulement deux variables sont nécessaires d'être définies
dans le fichier \code{.mk} ainsi qu'un appel à la macro de l'infrastructure
appropriée.

Maintenant, aller dans le \code{menuconfig}, activer {\em nInvaders}, et lancer
une compilation \code{make}. Vous devriez voir l'archive {\em nInvaders}
être téléchargée et extraite. Regarder dans \code{output/build/} pour voir
si ça a été extrait comme attendu.

\section{C'est parti pour la compilation !}

Comme vous avez pu le voir, {\em nInvaders} utilise un simple \code{Makefile}
pour son processus de build. Vous allez donc devoir définir les variables
de build pour {\em nInvaders}. Pour faire cela, 4 variables sont fournies par
Buildroot :

\begin{itemize}

\item \code{TARGET_MAKE_ENV}, qui devra être passé dans l'environnement utilisé
  lors de la commande \code{make}.

\item \code{MAKE}, contient le vrai appel du \code{make} avec potentiellement
  quelques paramètres pour paralleliser le build.

\item \code{TARGET_CONFIGURE_OPTS}, contient la définition de pleins de
  variables utiles dans les \code{Makefiles} comme : \code{CC},
  \code{CFLAGS}, \code{LDFLAGS}, etc.

\item \code{@D}, contient le chemin vers le répertoire où le code source de
  {\em nInvaders} a été extrait.

\end{itemize}

Quand vous créez des paquets Buildroot, c'est une bonne pratique de regarder
comment les autres paquets sont définis. Regarder par exemple le paquet
\code{jhead} qui sera très similaire à ce dont on a besoin pour notre paquet
\code{ninvaders}.

Lorsque vous avez écrit l'étape de build de {\em nInvaders}, il est temps
de tester! Mais si vous lancez seulement un \code{make} pour redémarrer
un build, le paquet \code{ninvaders} ne sera pas rebuildé car il l'a été
déjà été précemment.

On va donc forcer le build en supprimant complètement le répertoire des
sources :

\begin{verbatim}
make ninvaders-dirclean
\end{verbatim}

Ensuite, recommencer un build :

\begin{verbatim}
make
\end{verbatim}

Cette fois, vous devriez voir l'étape \code{ninvaders 0.1.1 Building} mais
elle échouera car le fichier \code{ncurses.h} est manquant.

\section{Gérer les dépendances}

Le fichier \code{ncurses.h} est manquant parce que {\em nInvaders}
dépends de la bibliothèque \code{ncurses} pour l'interface sur un terminal en
mode "texte". On va donc installer \code{ncurses} comme dépendance de
{\em nInvaders}. Pour cela :

\begin{itemize}

\item Indiquer la dépendance dans le fichier \code{Config.in}. Utiliser le
  \code{select} pour être sûr que le paquet \code{ncurses} sera automatiquement
  sélectionné avec la sélection de \code{ninvaders}.
\item Indiquer la dépendance dans le fichier \code{.mk}.

\end{itemize}

Recommencer un build du paquet avec :
\code{make ninvaders-dirclean all} (qui est pareil que
\code{make ninvaders-dirclean} suivi d'un \code{make}).

Maintenant, le paquet devrait être compilé avec succès !
Si vous jetez un oeil à \code{output/build/ninvaders-0.1.1/}, vous devriez
voir le binaire \code{nInvaders}. Lancer le programme \code{file} sur
\code{nInvaders} pour vérifier que c'est bien pour ARM.

\code{nInvaders} a bien été compilé mais il n'est pas pour autant installé
dans le rootfs de notre cible !

\section{Installation et test du programme}

Si vous regardez le \code{Makefile} des sources de {\em nInvaders},
vous remarquez qu'il n'y a rien pour installer le programme. Il n'y a pas
de règle \code{install:}.

Dans \code{ninvaders.mk}, on va devoir créer une {\em commande d'installation
  pour cible} qui va simplement installer le binaire \code{nInvaders}.
Utiliser la variable \code{$(INSTALL)} pour cela. Regarder le paquet
\code{jhead} comme exemple.

Rebuilder une nouvelle fois \code{ninvaders} et cette fois, vous devriez
voir le binaire \code{nInvaders} dans \code{output/target/usr/bin/}!

Reflasher votre rootfs sur la carte SD et rebooter le système.

\section{Ajout d'un fichier de hash}

Pour finaliser notre paquet, ajouter un fichier manquant : {\em hash file}
pour permettre de vérifier que les personnes builderont les mêmes sources
que celles que vous avez utilisé pour votre paquet. Pour savoir le hash,
SourceForge permet de directement connaitre cela via :
aller dans la page {\em download} et à coté du nom du fichier, il y a une
icône d'information qui vous fournit les hashes MD5 et SHA1.

Recompiler avec \code{make ninvaders-dirclean all}.
Regarder l'output du build et avant le message \code{ninvaders 0.1.1 Extracting},
vous devriez voir :

\begin{verbatim}
ninvaders-0.1.1.tar.gz: OK (sha1: ....)
ninvaders-0.1.1.tar.gz: OK (md5: ....)
\end{verbatim}

\section{Tester la suppression d'un paquet}

Pour jouer un peu avec Buildroot, réaliser les tests suivants : désactiver
le paquet \code{ninvaders} dans le \code{menuconfig} et redémarrer le build via
\code{make}. Quand le build est fini, regarder dans \code{output/target/}.
Est-ce que {\em nInvaders} est toujours installé ? Si oui, pourquoi ?
